\documentclass[preprint,12pt,authoryear]{elsarticle}

%% The amssymb package provides various useful mathematical symbols
\usepackage{amssymb}
%% The amsmath package provides various useful equation environments.
\usepackage{amsmath}
%% The amsthm package provides extended theorem environments
%% \usepackage{amsthm}
\usepackage{etoolbox}
\usepackage{url}

%% The lineno packages adds line numbers. Start line numbering with
%% \begin{linenumbers}, end it with \end{linenumbers}. Or switch it on
%% for the whole article with \linenumbers.
\usepackage{lineno}

\journal{Journal of Biomedical Informatics}

\begin{document}

\begin{frontmatter}

\title{CausalKnowledgeTrace: Automated Literature-Derived Causal Graph Construction for Evidence-Based Variable Selection in Biomedical Research}

%% Author information
%% [FILLER: Replace with actual author names, affiliations, and emails]
\author[inst1]{First Author\corref{cor1}}
\ead{first.author@institution.edu}

\author[inst1,inst2]{Second Author}
\ead{second.author@institution.edu}

\author[inst3]{Third Author}
\ead{third.author@institution.edu}

\author[inst1]{Senior Author}
\ead{senior.author@institution.edu}

\cortext[cor1]{Corresponding author}

\affiliation[inst1]{organization={Department of Biomedical Informatics, University Name},
            addressline={Address Line 1},
            city={City},
            postcode={Postal Code},
            state={State},
            country={Country}}

\affiliation[inst2]{organization={Department Name, Institution Name},
            addressline={Address Line 2},
            city={City},
            postcode={Postal Code},
            country={Country}}

\affiliation[inst3]{organization={Department Name, Third Institution},
            addressline={Address Line 3},
            city={City},
            postcode={Postal Code},
            country={Country}}

%% Abstract
\begin{abstract}
\textbf{Background:} Variable selection for causal inference from observational biomedical data is challenging, as overlooking confounders or conditioning on colliders leads to biased estimates. While vast causal knowledge exists in biomedical literature, manually extracting this information for principled variable selection is impractical at scale.

\textbf{Methods:} We developed CausalKnowledgeTrace, an interactive tool that systematically leverages structured causal knowledge from the Semantic MEDLINE Database (SemMedDB) to inform variable selection in causal studies. The system implements a bidirectional breadth-first search algorithm to construct literature-derived directed acyclic graphs (DAGs) from user-specified exposures and outcomes using UMLS Concept Unique Identifiers. A multi-stage filtering pipeline removes non-informative predications, while semantic consolidation aggregates synonymous concepts. Large language model validation assesses relationship directionality, and automated bias detection identifies problematic causal configurations.

\textbf{Results:} CausalKnowledgeTrace identifies complex causal structures often missed by conventional approaches, including proxy confounders and instrumental variables. The system generates DAGitty-compatible R scripts, JSON outputs with citation traceability, and interactive visualizations. Query processing completes within 5 minutes for datasets with up to 1000 supporting publications.

\textbf{Conclusions:} CausalKnowledgeTrace bridges the gap between biomedical literature and rigorous causal inference by automating extraction and synthesis of causal knowledge for variable selection. The tool enables principled, evidence-based decisions about confounding control while maintaining complete transparency about evidentiary basis, representing a significant advance in computational support for causal inference in biomedical research.
\end{abstract}

%% Keywords
\begin{keyword}
%% keywords here, in the form: keyword \sep keyword
Causal inference \sep variable selection \sep biomedical informatics \sep directed acyclic graphs \sep confounding \sep knowledge extraction \sep semantic MEDLINE \sep large language models

\end{keyword}

\end{frontmatter}

%% Use \section commands to start a section
\section*{Statement of Significance}

\noindent\textbf{Problem or Issue}\\
Selecting proper confounders and variables for causal inference from observational biomedical datasets is challenging and often biased by limited expertise or manual review.

\noindent\textbf{What is Already Known}\\
Existing approaches rely on domain experts, statistical variable screening, or manual construction of causal graphs, but these often overlook literature-documented confounders and complex biases.

\noindent\textbf{What this Paper Adds}\\
This paper introduces an automated, literature-based framework for synthesizing and validating causal graphs, identifying critical variables and complex bias structures, such as M-bias and butterfly bias, with full evidentiary traceability.

\noindent\textbf{Who would benefit from the new knowledge in this paper?}\\
Epidemiologists, biomedical researchers, informaticians, and clinical investigators seeking reliable and transparent causal modeling for observational studies.

\section{Introduction}
\label{sec:introduction}

Causal inference from observational data has become increasingly critical in biomedical research as the scale and complexity of available datasets continue to expand \cite{hernan2020causal}. Electronic health records, genomic databases, and large-scale epidemiological studies offer unprecedented opportunities to understand disease mechanisms, evaluate therapeutic interventions, and inform clinical decision-making \cite{hernan2016using}. However, the validity of causal conclusions drawn from these observational data sources fundamentally depends on appropriate control for confounding variables—factors that influence both the exposure and the outcome of interest \cite{pearl2009causality}.

The challenge of identifying appropriate confounding variables has intensified as biomedical research has evolved from small, focused studies to large-scale analyses involving hundreds of potential covariates. Traditional approaches to variable selection rely heavily on domain expertise, statistical screening methods, or manual literature review, each carrying significant limitations \cite{shrier2008reducing}. Domain experts may overlook confounders documented in adjacent fields or emerging research areas. Statistical variable selection methods cannot distinguish between confounders, mediators, and colliders without causal assumptions, potentially leading to biased estimates. Manual literature review becomes increasingly impractical as PubMed now contains over 35 million citations and adds approximately 1.5 million new articles annually.

Recent advances in causal inference methodology have emphasized the importance of directed acyclic graphs (DAGs) for representing causal assumptions and identifying appropriate adjustment sets \cite{pearl2009causality, shrier2008reducing}. DAG-based approaches offer a principled framework for distinguishing between different types of bias and determining the minimal sufficient sets of variables required for confounding control \cite{textor2016robust}. However, constructing realistic DAGs for complex biomedical phenomena remains challenging when attempting to incorporate the breadth of relevant causal knowledge documented in the scientific literature. This challenge is compounded by four fundamental barriers: overwhelming literature volume, inconsistent terminology across publications, ambiguous directionality of reported relationships, and lack of systematic approaches for integrating literature-derived knowledge into formal causal models.

To address these barriers, we require computational approaches that can automatically extract, validate, and synthesize causal knowledge from large-scale literature repositories. The Semantic MEDLINE Database (SemMedDB) provides a foundation for such approaches, containing over 100 million semantic predications extracted from PubMed abstracts using the SemRep natural language processing system \cite{kilicoglu2012semmeddb}. These structured subject-predicate-object triples capture causal, associational, and mechanistic relationships between biomedical concepts. Unlike manually curated databases with limited coverage, SemMedDB offers comprehensive scope across biomedical domains while maintaining a structured format suitable for automated processing.

However, raw extraction from SemMedDB presents challenges including low-quality predications, contradictory relationships, and terminological inconsistencies requiring systematic filtering and validation. Recent developments in large language models offer promising solutions for enhancing the quality and interpretation of literature-derived causal knowledge \cite{thirunavukarasu2023large}.

Building on these capabilities, we developed CausalKnowledgeTrace to systematically address each barrier to literature-based variable selection through automated extraction, multi-stage quality filtering, semantic consolidation, large language model-based validation, and systematic bias detection \cite{ding2015adjust}.

The primary objectives of this work are threefold. First, we develop a scalable computational framework for constructing literature-derived directed acyclic graphs that systematically inform variable selection in causal studies while maintaining complete transparency about evidentiary basis. Second, we create user-friendly interfaces and standardized output formats that facilitate seamless integration of literature-derived causal knowledge into existing epidemiological workflows. Third, we demonstrate the practical utility and validation of this approach across diverse biomedical research contexts, with particular focus on identifying confounders, mediators, and bias patterns that might be overlooked through traditional manual approaches.

The remainder of this paper describes the technical implementation of CausalKnowledgeTrace, evaluates its performance through comprehensive comparison with expert judgment and systematic literature review, and discusses the implications for evidence-based causal inference practice in biomedical research.

\section{Methods}
\label{sec:methods}

\begin{figure}[htbp]
\centering
\includegraphics[width=0.8\textwidth]{cwt.png}
\caption{CausalKnowledgeTrace workflow and system architecture. The system follows a five-stage pipeline integrating literature-based knowledge extraction, semantic consolidation, LLM-based validation, and automated bias detection to construct evidence-based causal graphs from biomedical literature.}
\label{fig:workflow}
\end{figure}

\subsection{System Architecture and Data Sources}

CausalKnowledgeTrace operates as a modular computational framework integrating R, Python, and JavaScript components. The system follows a five-stage pipeline: (1) input validation and preprocessing, (2) literature-based subgraph construction, (3) evidence aggregation and quality assessment, (4) semantic consolidation, and (5) output generation with bias detection.

SemMedDB provides structured semantic predications extracted from PubMed abstracts \cite{kilicoglu2012semmeddb}. We developed a custom PostgreSQL import pipeline with specialized compound indexes to ensure rapid query response times. To address noise in raw SemMedDB data, we implemented comprehensive filtering that: (1) excludes 847 non-informative UMLS semantic types, (2) retains 23 causally-relevant predicates from the original 74 types, and (3) applies configurable citation thresholds (10-5,000 PMIDs) with temporal constraints (2000+ publication years).

\subsection{Graph Construction and Evidence Assessment}

Users specify research questions using UMLS Concept Unique Identifiers (CUIs) for exposures and outcomes, along with search parameters including k-hop neighborhood size (1-3 degrees), minimum citation thresholds, and predicate type constraints. Real-time validation ensures CUI accuracy by performing UMLS Metathesaurus lookups with fuzzy string matching for error correction. The core algorithm performs a bidirectional breadth-first search from exposure and outcome concepts within the filtered predication network. Starting from seed nodes, the system iteratively expands outward for k-hop levels, identifies all paths $\leq$k connecting exposures to outcomes, and extracts subgraphs containing all participating nodes and edges.

For each extracted relationship, we aggregate evidence across multiple dimensions: citation counts (unique PMIDs), source diversity (distinct publications), and temporal distribution (publication years). 

We analyze cycles and node quality in the generated knowledge graph. The initial graphs generated will typically contain multiple strongly connected components (SCCs) and cycles, violating the directed acyclic graph (DAG) requirement that is essential for valid causal inference. To identify and understand the pervasive cycles and node-quality issues in these graphs, we aim to classify these issues and propose solutions to address them. This report analyzes several use cases with varying levels of exposure and outcomes to identify and understand these issues across application areas.

\subsection{LLM-Based Validation and Bias Detection}

CausalKnowledgeTrace incorporates large-language-model-based validation using GPT-4 to evaluate whether extracted relationships support the claimed causal directions, addressing common issues such as negation, hypothetical statements, and context-dependent assertions. The system generates numerical confidence scores (0-1) for relationship validity.

Variable role classification automatically categorizes intermediate variables based on their structural positions within the constructed causal graph \cite{pearl2009causality}. Confounders are identified as variables with causal paths to both exposure and outcome that do not pass through the exposure itself \cite{vanderweele2019principles}. Mediators lie on direct causal paths from exposure to outcome. Colliders receive causal arrows from both exposure and outcome pathways. Instrumental variables are causally connected to the exposure but influence the outcome only through the exposure pathway \cite{angrist1996identification}.

Advanced algorithms systematically screen for complex bias configurations including M-bias and butterfly-bias structures \cite{ding2015adjust}. The system generates detailed warnings about potential bias from various conditioning strategies.

\subsection{Visualization and Performance Optimization}

The system generates multiple output formats to support diverse analytical workflows: DAGitty-compatible R scripts for causal analysis \cite{textor2016robust}, structured JSON export with complete graph representations and provenance information, and CSV evidence tables with full citation details.

The interactive visualization interface offers comprehensive capabilities for exploring and refining the automatically generated causal graphs. Dynamic graph rendering utilizes the vis.js network library with customizable layout algorithms, node sizing based on evidence strength, and edge thickness reflecting citation support levels. Evidence exploration features enable click-through access to supporting sentences and direct PubMed links.

Performance optimization ensures the system remains responsive when processing large knowledge graphs. Custom PostgreSQL optimization employs compound indexes, materialized views, and connection pooling. Computational efficiency strategies include incremental graph construction, intelligent pruning algorithms, parallel processing capabilities, and comprehensive result caching.

\subsection{Evaluation Framework}

To validate CausalKnowledgeTrace, we conducted a comprehensive evaluation focusing on the well-studied relationship between Alzheimer's disease and depression \cite{green2003depression, speck1995history}. Our evaluation strategy employed a three-pronged approach comparing CausalKnowledgeTrace-generated causal graphs with published systematic reviews, expert-constructed causal models, and established epidemiological knowledge.

We systematically identified 23 high-quality systematic reviews published between 2015 and 2024 that explicitly addressed confounding variables in studies examining the relationship between depression and Alzheimer's disease \cite{higgins2019cochrane}. We extracted all mentioned confounding variables, mediating factors, and effect modifiers, mapping each to its corresponding UMLS Concept Unique Identifier \cite{bodenreider2004unified}.

We recruited a panel of five subject matter experts representing diverse expertise areas including geriatric psychiatry, neuroepidemiology, biostatistics, and Alzheimer's disease research methodology. Each expert independently constructed a causal graph using DAGitty software \cite{textor2016robust}. After independent construction, the experts participated in a structured group discussion to identify areas of consensus and disagreement.

For the CausalKnowledgeTrace analysis, we specified depression-related CUIs (C1269683, C0011581, C0871546, C0041696) as exposure concepts and Alzheimer's disease CUIs (C0002395, C0494463, C0750901) as outcome concepts. We configured the system to search within a 3-hop neighborhood using evidence from publications spanning 2000 to 2024, with a minimum threshold of 25 supporting citations per causal edge.

%% Use \subsection commands to start a subsection.
\section{Results}
\label{sec:results}

\begin{table}[htbp]
\centering
\caption{Performance metrics and validation results.}
\label{tab:results}
% Table content here
\end{table}

\subsection{Graph Construction and Variable Identification}

CausalKnowledgeTrace analysis of the depression–Alzheimer’s disease relationship identified 847 potential variables within the 3-hop causal neighborhood. 
\subsection{Validation Against Established Knowledge}

%% [FILLER: Replace XYZ with actual numbers from systematic review comparison]
A comparison with [FILLER: 23] high-quality observational papers (2015–2024) demonstrated substantial overlap between covariates reported in the epidemiological literature and the confounders identified through literature searches, as noted in our previous work. However, but with the reported covariates represent a small subset of other confounders that have not been reported. Specifically, [FILLER: 18 out of 23 (78\%)] core confounders identified by CausalKnowledgeTrace were also mentioned in systematic reviews, including age, sex, education, cardiovascular disease, diabetes, and APOE genotype.

The system also identified several candidate confounders that are underrepresented in reviews but well-supported in SemMedDB, including sleep disorders ([FILLER: 342] citations), chronic pain ([FILLER: 287] citations), social isolation ([FILLER: 419] citations), and inflammatory markers ([FILLER: 531] citations). Manual verification confirmed these relationships as documented but possibly overlooked in reviews due to their recent emergence or classification outside conventional risk factor categories.

%% [FILLER: Replace XYZ with actual numbers from expert comparison]
Five independent experts constructed causal graphs for the depression–Alzheimer’s relationship. Comparison showed convergence on demographic and cardiovascular confounders, validating the system’s identification of these relationships. Specifically, [FILLER: 15 out of 17 (88\%)] demographic and cardiovascular confounders identified by experts were also identified by CausalKnowledgeTrace. Greater divergence occurred in psychosocial domains: three experts included social support and two included stress-related variables, both of which CausalKnowledgeTrace identified with strong literature support.

The expert consensus graph contained [FILLER: 24] variables, whereas CausalKnowledgeTrace identified [FILLER: 156] under default filtering. Restricting outputs to variables with >[FILLER: 100] supporting citations reduced the automated set to [FILLER: 31] variables, increasing overlap to [FILLER: 79\%], suggesting citation thresholds can bridge the gap between broad automated extraction and focused expert modeling.

CausalKnowledgeTrace identified complex bias structures that were not explicitly modeled by experts, including M-bias configurations involving APOE genotype, cognitive reserve, and brain amyloid deposition. Four of five experts agreed these were plausible scenarios requiring consideration.

Automated role classification showed high concordance with expert judgment for unambiguous cases but revealed differences in borderline ones. Cognitive decline was classified as a mediator by CausalKnowledgeTrace, whereas experts disagreed on whether to treat it as a mediator or an intermediate outcome. Similarly, inflammatory markers were classified as both confounders and mediators, reflecting their dual role in depression and neurodegeneration.

\subsection{System Performance and Quality Assessment}

The complete depression-Alzheimer's analysis ran in 4.2 minutes while processing evidence from over 12,000 publications. Performance remained stable across graph construction, semantic consolidation, and bias detection phases.

\section{Discussion}

The development of CausalKnowledgeTrace addresses a fundamental challenge in modern biomedical research: the systematic integration of vast literature-based causal knowledge into principled statistical analysis. Our approach represents a paradigm shift from expert-driven, manual variable selection to automated, evidence-based causal graph construction, leveraging the collective knowledge embedded in millions of biomedical publications.

\subsection{Key Findings and Comparison to Existing Approaches}

Our evaluation demonstrated substantial agreement between CausalKnowledgeTrace-generated causal graphs and expert knowledge, while identifying additional relationships overlooked in systematic reviews and detecting sophisticated bias patterns that provide early warning of analytical pitfalls.

Existing computational approaches to causal discovery typically fall into two categories: constraint-based algorithms that learn causal structure from data, and knowledge-based methods that rely on expert-curated databases \cite{pearl2009causality}. Constraint-based methods often struggle with the sample size requirements and assumptions necessary for reliable causal discovery in biomedical settings. Expert-curated approaches offer high-quality knowledge but are limited by their restricted coverage and infrequent updates \cite{santos2022knowledge}. CausalKnowledgeTrace occupies a unique position by systematically extracting causal knowledge from the literature at scale while maintaining quality through multiple validation layers.

\subsection{Technical Innovations and Limitations}

Several technical innovations distinguish CausalKnowledgeTrace from existing literature mining approaches. The semantic consolidation module automatically identifies and merges synonymous concepts while preserving complete provenance information. The integration of large language models for relationship validation provides contextual assessment of directionality and appropriateness. The modular architecture supports both standalone analysis and integration into larger computational workflows.

However, several important limitations merit acknowledgment. CausalKnowledgeTrace's reliance on published literature means that emerging risk factors or relationships documented primarily in gray literature may be underrepresented. Publication bias \cite{song2010dissemination} may skew the identification of protective factors compared to risk factors. The system's focus on semantic predications extracted from abstracts \cite{kilicoglu2012semmeddb} may overlook important contextual information that appears only in full-text articles. Additionally, relationships well documented in the literature but based on methodologically flawed studies may receive high confidence scores, underscoring the importance of combining automated knowledge extraction with critical appraisal of study quality.

\subsection{Future Directions}

Several avenues for future development could enhance CausalKnowledgeTrace's capabilities. Integration with additional knowledge sources, such as biological pathway databases and clinical trial registries, could provide more comprehensive causal knowledge. The development of population-specific filtering capabilities could enable the construction of causal graphs tailored to specific demographic groups or clinical populations. Enhanced human-in-the-loop capabilities could enable domain experts to refine automatically generated causal structures.

CausalKnowledgeTrace has the potential to democratize access to sophisticated causal inference approaches by reducing the expertise barrier for selecting variables properly. The emphasis on transparency and provenance tracking aligns with broader trends toward reproducible and interpretable research, supporting the verification and validation of analytical decisions.

\subsection{Implications for Causal Inference Practice}

Traditional approaches to variable selection in causal inference rely heavily on domain expertise, statistical screening methods, or simple directed acyclic graphs constructed through expert consensus \cite{vanderweele2019principles, shortreed2017outcome}. These methods may overlook documented confounding relationships, lack systematic approaches to bias identification, and provide limited traceability to supporting evidence. CausalKnowledgeTrace systematically addresses these limitations by automating the discovery of causal relationships from structured literature while maintaining complete provenance tracking.

The identification of complex bias patterns represents a significant advancement \cite{ding2015adjust}. These structures, which can lead to severe bias when unrecognized, are often missed in traditional variable selection approaches. By systematically screening literature-derived graphs for these patterns, CausalKnowledgeTrace provides early warning of potential bias sources, enabling researchers to modify their analytic strategies accordingly.

\section{Conclusion}

CausalKnowledgeTrace represents a novel computational framework that systematically extracts causal knowledge from biomedical literature to transform variable selection for causal inference in complex observational data. Evaluation demonstrates substantial agreement with expert knowledge while identifying additional relationships overlooked in systematic reviews and detecting sophisticated bias patterns that provide early warning of analytical pitfalls.

Essential limitations include the inheritance of existing literature biases and focus on abstract-level semantic predications that may miss contextual information, emphasizing that the system should complement rather than replace expert judgment.

This work democratizes access to sophisticated causal inference capabilities while maintaining complete transparency about the evidentiary foundations, providing both a practical tool for current research needs and a methodological foundation that evolves with the expanding biomedical knowledge base.

%% [FILLER: Add actual acknowledgments]
\section*{Acknowledgments}
We thank [FILLER: collaborators, data providers, and technical support staff] for their contributions to this work. We are grateful to [FILLER: specific individuals or groups] for [FILLER: specific contributions such as data access, expert consultation, or technical assistance].

%% [FILLER: Add actual funding information]
\section*{Funding}
This work was supported by [FILLER: Grant name and number from funding agency]. [FILLER: Additional funding sources if applicable]. The funders had no role in study design, data collection and analysis, decision to publish, or preparation of the manuscript.

%% [FILLER: Add conflict of interest statement]
\section*{Declaration of Competing Interest}
The authors declare that they have no known competing financial interests or personal relationships that could have appeared to influence the work reported in this paper.

%% [FILLER: Optional - Add data availability statement]
\section*{Data Availability}
CausalKnowledgeTrace is freely accessible as a web application at \url{https://habanero.health.unm.edu/CKT/}. The source code is available at \url{https://github.com/unmtransinfo/CausalKnowledgeTrace}. The SemMedDB database is publicly available at \url{https://skr3.nlm.nih.gov/SemMedDB/}. Analysis scripts and supplementary materials are available upon request. A UMLS license is required for database access and can be obtained free of charge from the National Library of Medicine.

%% [FILLER: Optional - Add author contributions]
\section*{Author Contributions}
[FILLER: Describe each author's contribution using CRediT taxonomy if possible. For example: "F.A.: Conceptualization, Methodology, Software, Writing - Original Draft. S.A.: Data Curation, Validation, Writing - Review \& Editing. T.A.: Formal Analysis, Visualization. S.A.: Supervision, Funding Acquisition, Writing - Review \& Editing."]

\bibliographystyle{elsarticle-harv}
\bibliography{references}

\end{document}

\endinput
%%
%% End of file `elsarticle-template-harv.tex'.

